\documentclass[12pt]{article}

%Packages BEGIN

%Packages END

\title{Data Structures PA2}
\author{{\scriptsize B07202020}\\Hao-Chien Wang}

\begin{document}
\maketitle

\section{Usage}%
Use \texttt{python ./hash.py <password file>} to create \texttt{./Dictionary.txt}.
Then, use \texttt{python ./search.py} to search the hash values in 
\texttt{./list\_pa2.txt} and write the results in \texttt{./results\_pa2.txt}.
Alternatively, use \texttt{python ./manual\_search.py} to search one by one.

\section{Code Structure}%

The code generating the dictionary is written in \texttt{./hash.py}, and the
code to search for a given hash value is defined in \texttt{./search.py}.
\texttt{./manual\_search.py} imports the function defined in \texttt{./search.py}
to perform the same functionality.

\section{Data Structures and Algorithms}%

The algorithm used in this program is straightforward. In \texttt{./hash.py},
passwords are read one by one, the hash value is calculated for each salt and
written to the output file. In \texttt{./search.py}, the aforementioned output
file is read and a table consist of password, salt and hash values is created 
(in the form of Python lists). For each hash values, linear search is performed
on the table and the output is written. In \texttt{./manual\_search.py}, 
the same procedure is performed, but on the input read from commandline instead
of a file.

\end{document}
